\documentclass{beamer}
\usepackage{tikz}
\usepackage{tkz-euclide} % loads  TikZ and tkz-base
\usetkzobj{all}
\usetikzlibrary{calc,math}
\usepackage[utf8]{inputenc}
\usepackage{amsmath}
\usetheme{Boadilla}
\title{My Presentation}
\author{Siddhartha Pothukuchi}
\institute{Indian Institute of Technology, Bhilai.}
\date{\today}
\begin{document}


\begin{frame}
\titlepage
\end{frame}
\section{Question}
\begin{frame}
\frametitle{Question}
\begin{block}{Exercise 8.1(Q no.28)}
In right triangle ABC, right angled at C, M is
the mid-point of hypotenuse AB. C is joined to
M and produced to a point D such that DM =
CM. Point D is joined to point B. Show that:
\newline
\hyperlink{a}{\beamerbutton{a)$\triangle  AMC  \cong   \triangle  BMD $}}
\newline
\hyperlink{b}{\beamerbutton{b)$\triangle DBC $ is a right angle.}}
\newline
\hyperlink{c}{\beamerbutton{c)$\triangle  DBC  \cong  \triangle  ABC $}}
\newline
\hyperlink{d}{\beamerbutton{d)CM = $\frac{1}{2}$ AB}}

\end{block}
\end{frame}

\section{Solution}
\subsection{a}
\begin{frame}
\begin{figure}
\begin{block}{Codes}
The python code for the figure is $/code/traingle.py$
\newline
The latex- tikz code is $/figs/triangle.tex$
\newline
The above latex code can be compiled as standalone document $/figs/triangle\_fig.tex$
\end{block}.
\input{./figs/triangle.tex}
\caption{Right Angled Triangle}
\end{figure}
\end{frame}
\begin{frame}
\frametitle{Solution a)}
\label{a}
\small
From the above figure,
$$
C=
\begin{pmatrix}
0\\
0
\end{pmatrix}
,A=
\begin{pmatrix}
0\\
b
\end{pmatrix}
,B=
\begin{pmatrix}
a\\
0
\end{pmatrix}
$$
As, M is the midpoint of AB
$$
M=
\begin{pmatrix}
a/2 \\
b/2 
\end{pmatrix}
$$
Therefore Coordinates of D are
$$
D=
\begin{pmatrix}
a\\
b
\end{pmatrix}
$$
\newline
$\triangle AMC$ and $\triangle DMB$ are congruent to each other by SAS congruency.
\newline
(i) Side AM  is equal to the corresponding side BM  [As M is midpoint of AB]
\newline
(ii)Side CM of is equal to corresponding side DM [As M is midpoint of DC]
\newline
(iii)$\angle AMC$ = $\angle DMB$ [ Vertically Opposite Angles]
\end{frame}
\subsection{b}
\begin{frame}
\frametitle{Solution b)}
\label{b}
\small
In $\triangle ACB$ \[(AB)^2=a^2+b^2\]
Since $\angle ACB$ = 90$^{\circ}$[ Pythagorus theorem]
\newline
In $\triangle DBC$ 
\begin{block}{Formula}
cos $\angle DBC$= \[((a^2+b^2-(DC)^2)/2ab)\] [DB = D-B = b]
\end{block}
By using distance formula i.e $\sqrt{(x1-x2)^2+(y1-y1)^2}$ we get that AB=DC from the given coordinates.
\newline
$cos\angle DBC$ =\[((a^2+b^2-(AB)^2)/2ab)\]
$cos\angle DBC$=0
\newline
Therefore, $\angle DBC$ is right angle
\end{frame}
\subsection{c}
\begin{frame}
\frametitle{Solution c)}
\label{c}
$\triangle ACB$ and $\triangle DCB$ are congruent to each other in SAS congruency.
(i)Both the triangles have a common base , a.
\newline
(ii)AC = DB by using distance formula
\newline
(iii)$\angle ACB$ = $\angle DBC$ = 90$^{\circ}$ [From \hyperlink{b}{\beamerreturnbutton{Solution b}}]

\end{frame}
\subsection{d}
\begin{frame}
\frametitle{Solution d)}
\label{d}
Since M is the midpoint of CD
\newline
CM=$\frac{1}{2}$ DC
From \hyperlink{b}{Solution b} it is clear that DC=AB
\newline
Therefore CM=$\frac{1}{2}$AB
\newline
Hence Proved.

\end{frame}







\end{document}
