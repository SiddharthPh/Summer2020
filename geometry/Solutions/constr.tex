
\begin{flushleft}
The python code for the figure is
\begin{lstlisting}
./code/traingle.py
\end{lstlisting}
The latex- tikz code is
\begin{lstlisting}
./figs/triangle.tex
\end{lstlisting}
The above latex code can be compiled as standalone document
\begin{lstlisting}
./figs/triangle_fig.tex
\end{lstlisting}
\end{flushleft}
%\begin{columns}
%\column{0.5\textwidth}
\begin{figure}[H]
\includegraphics[scale=0.4]{./figs/triangle.eps}
\caption*{a) By Python}
%
\input{./figs/triangle.tex}
\caption*{b) By Latex-tikz}
%
\end{figure}
The tables below are the values used for constructing the triangles in both Python and Latex-Tikz.
\begin{table}[H]
\centering
\begin{tabular}{ |p{3cm}|p{3cm}|  }
\hline
 \multicolumn{2}{|c|}{Initial Input Values.} \\
\hline
a & 4\\
\hline
b & 3\\
\hline
$\angle(ACB)$ & $90^{\circ}$ \\
\hline
\end{tabular}
\caption{To construct $\triangle ACB$}
\end{table}
The steps for constructing $\triangle ACB$ are
\newline
(i) Let$$ \vec{C}= \begin{pmatrix}0\\0\end{pmatrix}$$
\newline
(ii)$$\vec{A}=\begin{pmatrix}0\\3\end{pmatrix}$$
\\
(iii)$$\vec{B}=\begin{pmatrix}4\\0\end{pmatrix}$$
\\
Since, $\vec{M}$ is the midpoint of $\vec{AB}$ and $\vec{CD}$
\\
$$\vec{M}=(1/2)(\vec{A}+\vec{B})$$
\\
$$\vec{M}=\begin{pmatrix}2\\1.5\end{pmatrix}$$
\\
$$\vec{D}=2\vec{M}$$
\\
$$\vec{D}=\begin{pmatrix}4\\3\end{pmatrix}$$
\begin{table}[H]
\centering
\begin{tabular}{ |p{3cm}|p{3cm}|  }
\hline
 \multicolumn{2}{|c|}{Derived Values.} \\
\hline
$\vec{M}$ & $$\begin{pmatrix}2\\1.5\end{pmatrix}$$\\
							
\hline
$\vec{D}$ & $$\begin{pmatrix}4\\3\end{pmatrix} $$\\
\hline
\end{tabular}
\caption{To construct $\triangle DCB$}
\end{table}
