
\begin{flushleft}
The python code for the figure is
\begin{lstlisting}[frame=single]
./code/traingle.py
\end{lstlisting}
The latex- tikz code is
\begin{lstlisting}[frame=single]
./figs/triangle.tex
\end{lstlisting}
The above latex code can be compiled as standalone document
\begin{lstlisting} [frame=single]
./figs/triangle_fig.tex
\end{lstlisting}
\end{flushleft}
%\begin{columns}
%\column{0.5\textwidth}
\begin{figure}[H]
\includegraphics[scale=0.4]{./figs/triangle.eps}
\caption*{a) By Python}
%
\input{./figs/triangle.tex}
\caption*{b) By Latex-tikz}
%
\end{figure}
The tables below are the values used for constructing the triangles in both Python and Latex-Tikz.
\begin{table}[H]
\centering
\begin{tabular}{ |p{3cm}|p{3cm}|  }
\hline
 \multicolumn{2}{|c|}{Initial Input Values.} \\
\hline
$\vec{BC}(a)$ & $4\hat{i}$\\
\hline
$\vec{AC}(b)$ & $3\hat{j}$\\
\hline
$\angle(ACB)$ & $90^{\circ}$ \\
\hline
\end{tabular}
\caption*{To construct $\triangle ACB$}
\end{table}

\begin{table}[H]
\centering
\begin{tabular}{ |p{3cm}|p{3cm}|  }
\hline
 \multicolumn{2}{|c|}{Derived Values.} \\
\hline
$\vec{CM}$ & $2\hat{i}+1.5\hat{j}$\\
\hline
$\vec{CD}$ & $4\hat{i}+3\hat{j}$\\
\hline
\end{tabular}
\caption*{To construct $\triangle DCB$}
\end{table}