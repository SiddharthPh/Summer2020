\documentclass[journal,12pt,twocolumn]{IEEEtran}
\usepackage{listings}
\usepackage{ragged2e}
\usepackage[utf8]{inputenc}
\usepackage{subcaption}
\usepackage{tikz}
\usepackage[export]{adjustbox}
\usepackage{tkz-euclide} % loads  TikZ and tkz-base
%\usetkzobj{all}
\usetikzlibrary{calc,math}
\usepackage{amsmath}
\lstset{
%language=C,
frame=single, 
breaklines=true,
columns=fullflexible
}
\usepackage{float}
\newcommand\norm[1]{\left\lVert#1\right\rVert}
\title{Document on Question 28 Exercise(8.1)}
\author{Pothukuchi Siddhartha}

\begin{document}

\maketitle
\begin{abstract}
This a simple document explaining a question about the concept of similar triangles.
\end{abstract}
Download all python codes from 
%


\begin{lstlisting}
svn co https://github.com/SiddharthPh/Summer2020/trunk/document/codes
\end{lstlisting}
%

\section*{\textbf{Question}}
In right triangle ABC, right angled at C, M is
the mid-point of hypotenuse AB. C is joined to
M and produced to a point D such that DM =
CM. Point D is joined to point B. Show that:
\newline
a)$\triangle  AMC  \cong   \triangle  BMD $
\newline
b)$\triangle DBC $ is a right angle.
\newline
c)$\triangle  DBC  \cong  \triangle  ABC $
\newline
d)CM = $\frac{1}{2}$ AB

\section*{\textbf{Construction}}

\begin{flushleft}
The python code for the figure is
\begin{lstlisting}[frame=single]
./code/traingle.py
\end{lstlisting}
The latex- tikz code is
\begin{lstlisting}[frame=single]
./figs/triangle.tex
\end{lstlisting}
The above latex code can be compiled as standalone document
\begin{lstlisting} [frame=single]
./figs/triangle_fig.tex
\end{lstlisting}
\end{flushleft}
%\begin{columns}
%\column{0.5\textwidth}
\begin{figure}[H]
\includegraphics[scale=0.4]{./figs/triangle.eps}
\caption*{a) By Python}
%
\input{./figs/triangle.tex}
\caption*{b) By Latex-tikz}
%
\end{figure}
The tables below are the values used for constructing the triangles in both Python and Latex-Tikz.
\begin{table}[H]
\centering
\begin{tabular}{ |p{3cm}|p{3cm}|  }
\hline
 \multicolumn{2}{|c|}{Initial Input Values.} \\
\hline
$\vec{BC}(a)$ & $4\hat{i}$\\
\hline
$\vec{AC}(b)$ & $3\hat{j}$\\
\hline
$\angle(ACB)$ & $90^{\circ}$ \\
\hline
\end{tabular}
\caption*{To construct $\triangle ACB$}
\end{table}

\begin{table}[H]
\centering
\begin{tabular}{ |p{3cm}|p{3cm}|  }
\hline
 \multicolumn{2}{|c|}{Derived Values.} \\
\hline
$\vec{CM}$ & $2\hat{i}+1.5\hat{j}$\\
\hline
$\vec{CD}$ & $4\hat{i}+3\hat{j}$\\
\hline
\end{tabular}
\caption*{To construct $\triangle DCB$}
\end{table}
\section*{\textbf{Solution}}

From the figure, lets assume $\vec{C}$ to be the origin.
\newline
\begin{figure}[H]
\input{./figs/triangleABC.tex}
\caption{$\triangle ACB$}
\end{figure}
$\vec{C}=0$
\newline
$\norm{\vec{CA}}$=b
\newline
$\norm{\vec{CB}}$=a
\newline
$\vec{M}$ is the position vector of mid-point of $\vec{BA}$.
\newline
$\vec{CM} = \vec{CB}+\vec{BM}$ [$\vec{BM}=(1/2)*\vec{BA}$]
\newline
$$\vec{CM} =\begin{pmatrix}a\\0\end{pmatrix}+\begin{pmatrix}-a\\b/2\end{pmatrix}$$
\newline
Therefore, $$\vec{CM}=\begin{pmatrix}a/2\\b/2\end{pmatrix}$$
\newline
\begin{figure}[H]
\input{./figs/triangleDBC.tex}
\caption{$\triangle DBC$}
\end{figure}
From the figure, $\vec{CD}=2(\vec{CM})$
\newline
$$\vec{CD}=\begin{pmatrix}a\\b\end{pmatrix}$$

\subsection*{\textbf{Sol.a)}}
\input{./Solutions/sol_a.tex}
\subsection*{\textbf{Sol.b)}}
\input{./Solutions/sol_b.tex}
\subsection*{\textbf{Sol.c)}}
\input{./Solutions/sol_c.tex}
\subsection*{\textbf{Sol.d)}}
\input{Solutions/sol_d.tex}

\end{document}