\renewcommand{\theequation}{\theenumi}
\begin{enumerate}[label=\thesection.\arabic*.,ref=\thesection.\theenumi]
\numberwithin{equation}{enumi}
\item The Figure of the quadriletral as obtained in the question looks like Fig. \ref{fig:quadrilateral}.
with angles $\phase{ A},\phase{ C}$ and $\phase{ B}$ and$\phase{D}$ and sides $a, b$ and $c$ and $d$.


%\renewcommand{\thefigure}{\theenumi.\arabic{figure}}
\begin{figure}[!ht]
\centering
\resizebox{\columnwidth}{!}{\input{./figs/quad.tex}}
\caption{Quadrilateraal by Latex-Tikz}
\label{fig:quadrilateral}	
\end{figure}
%
%
%\renewcommand{\thefigure}{\theenumi}
%
\item The design parameters for construction are:
\label{const:table1}
\\
\solution See Table. \ref{table:table1}. 
%
\begin{table}[ht!]
\centering
%\begin{tabular}{ |p{3cm}|p{3cm}|  }
%\hline
% \multicolumn{2}{|c|}{Initial Input Values.} \\
%\hline
%a & 4\\
%\hline
%b & 3\\
%\hline
%$\phase{(ACB)$ & $90^{\circ}$ \\
%\hline
%\end{tabular}
\input{./tables/inp.tex}
\caption{To construct Quadrilateral ABCD}
\label{table:table1}	
\end{table}

%\item
%	For simplicity, let the greek letter $\alpha = \phase{ B$.  We have the following definitions.
%\begin{equation}
%\label{eq:tri_trig_defs}
%\begin{matrix}
	%\sin \theta = \frac{b}{c} & 	\cos \theta = \frac{a}{c} \\
	%\tan \theta = \frac{c}{a} & \cot \theta = \frac{1}{\tan \theta} \\
	%\csc \theta = \frac{1}{\sin \theta} & \sec \theta = \frac{1}{\cos \theta}
	%\end{equation}
%
\item Find the angular bisectors of each angle in Fig. \ref{fig:quadrilateral}
\label{const:quadrilateral}
\\
%
\solution 
From the given information, the line equation of acute angular bisector of $\phase B$ in vector form is 
%$\triangle ABC$ are 
\begin{align}
\label{eq:constr_a}
\vec{L1} =  \vec{B}+s(\vec{R1})
\\
\vec{L1} = \myvec{0\\0}+s\myvec{1.6\\0.8}
\end{align}
Where $\vec{R1}$ is the direction ration of the line $\vec{L1}$ obtained by the formula
\\
$$\vec{R1}=\frac{\vec{A}-\vec{B}}{\norm{A-B}}+\frac{\vec{C}-\vec{B}}{\norm{C-B}}$$
\\
Vector form of angular bisector of $\phase C$ is
\begin{align}
\label{eq:constr_b}
\vec{L2} = \vec{C}+t(\vec{R2})
\\
\vec{L2} = \myvec{9\\0}+t\myvec{-1.316\\0.948}
\end{align}
Where $\vec{R2}$ is the d.r of the line $\vec{L2}$ obtained by the formula
\\
$$\vec{R2} = \frac{\vec{A}-\vec{B}}{\norm{A-B}}+\frac{\vec{C}-\vec{B}}{\norm{C-B}}$$
\\
Vector form of angular bisector of $\phase A$ is
\begin{align}
\label{eq:constr_c}
\vec{L3} = \vec{A}+u(\vec{R3})
\\
\vec{L3} = \myvec{3\\4}+u\myvec{0.294\\-0.352}
\end{align}
Where $\vec{R3}$ is the d.r of the line $\vec{L3}$ obtained by the formula
\\
$$\vec{R3} = \frac{\vec{B}-\vec{C}}{\norm{B-C}}+\frac{\vec{D}-\vec{C}}{\norm{D-C}}$$
\\
Vector form of angular bisector of $\phase D$ is
\begin{align}
\label{eq:constr_d}
\vec{L4} = \vec{D}+v(\vec{R4})
\\
\vec{L4} = \myvec{7\\6}+v\myvec{-0.578\\-1.395}
\end{align}
Where $\vec{R4}$ is the d.r of the line $\vec{L4}$ obtained by the formula
\\
$$\vec{R4} = \frac{\vec{A}-\vec{D}}{\norm{A-D}}+\frac{\vec{C}-\vec{D}}{\norm{C-D}}$$
\\ 
Here s,t,u,v are constants used to define a line in vector form, where a unique position vector is obtained for unique values of (s,t,u,v) of the respective line.
\\
\item To find the point of intersection of the angular bisectors, equate the respective line equations.
\solution
$\vec{E}$ is obtained by equating line equations $\vec{L1}$ and $\vec{L2}$
\begin{align}
\myvec{1.6s\\0.8s}=\myvec{9-1.316t\\0.948t}
\\
\myvec{1.6s+1.316t-9\\0.8s-0.948t}=\myvec{0\\0}
\end{align}
By solving the two equations we obtain the values of s,t.
\\
By substituting the values in $\vec{L1}$ we obtain $\vec{E}$
\\
$$\vec{E}=\myvec{5.3137\\2.6568}$$
\\
$\vec{F}$ is obtained by equating line equations $\vec{L1}$ and $\vec{L3}$
\begin{align}
\myvec{1.6s\\0.8s}=\myvec{3+0.294u\\4-0.352u}
\\
\myvec{1.6s-0.294u-3\\0.8s+0.352u-4}=\myvec{0\\0}
\end{align}
By solving the two equations we obtain the values of s,u.
\\
By substituting the values in $\vec{L1}$ we obtain $\vec{F}$
\\
$$\vec{F}=\myvec{4.472\\2.236}$$
\\
$\vec{G}$ is obtained by equating line equations $\vec{L3}$ and $\vec{L4}$
\begin{align}
\myvec{3+0.294u\\4-0.352u}=\myvec{7-0.578v\\6-1.395v}
\\
\myvec{0.294u+0.578v-4\\-0.352u+1.395v-2}=\myvec{0\\0}
\end{align}
By solving the two equations we obtain the values of u,v.
\\
By substituting the values in $\vec{L3}$ we obtain $\vec{G}$
\\
$$\vec{G}=\myvec{5.119\\1.460}$$
\\
$\vec{H}$ is obtained by equating line equations $\vec{L2}$ and $\vec{L4}$
\begin{align}
\myvec{9-1.316t\\0.948t}=\myvec{7-0.578v\\6-1.395v}
\\
\myvec{-1.316t+0.578v+2\\0.948t+1.395v-6}=\myvec{0\\0}
\end{align}
By solving the two equations we obtain the values of t,v.
\\
By substituting the values in $\vec{L2}$ we obtain $\vec{H}$.
\\
$$\vec{H}=\myvec{5.545\\2.489}$$
The values are listed in 
%\item List the  derived values.
%\label{const:table2}
%\\
%\solution See  
Table. \ref{table:table2} 
\begin{table}[ht!]
\centering
%\begin{tabular}{ |p{3cm}|p{3cm}|  }
%\hline
% \multicolumn{2}{|c|}{Derived Values.} \\
%\hline
%$\vec{M}$ & $$\begin{pmatrix}2\\1.5\end{pmatrix}$$\\						
%\hline
%$\vec{D}$ & $$\begin{pmatrix}4\\3\end{pmatrix} $$\\
%\hline
\input{./tables/inp1.tex}
\caption{Cyclic Quadrilateral EFGH}
\label{table:table2}

\end{table}
%
\item Draw Fig. \ref{fig:quadrilateral}.	
\\
\solution The  following Python code generates Fig. \ref{fig:quadri_py}
%
\begin{lstlisting}
codes/quad.py
\end{lstlisting}
\begin{figure}[!ht]
\centering
\includegraphics[width=\columnwidth]{./figs/quad.eps}
\caption{Quadrilateral generated using python}
\label{fig:quadri_py}
\end{figure}

%
and the equivalent latex-tikz code generating Fig. \ref{fig:quadrilateral} is 
\begin{lstlisting}
figs/quad.tex
\end{lstlisting}
%
The above latex code can be compiled as a standalone document as
\begin{lstlisting}
figs/quad_fig.tex
\end{lstlisting}

%

%

%
%

\end{enumerate}
