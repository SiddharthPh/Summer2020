\renewcommand{\theequation}{\theenumi}
\begin{enumerate}[label=\arabic*.,ref=\thesubsection.\theenumi]
\numberwithin{equation}{enumi}
\item The Figure of the quadriletral as obtained in the question looks like Fig. \ref{fig:quadrilateral}.
with angles $\phase{ A},\phase{ C}$ and $\phase{ B}$ and$\phase{D}$ and sides $a, b$ and $c$ and $d$.


%\renewcommand{\thefigure}{\theenumi.\arabic{figure}}
\begin{figure}[!ht]
\centering
\resizebox{\columnwidth}{!}{\input{./figs/quad.tex}}
\caption{Quadrilateraal by Latex-Tikz}
\label{fig:quadrilateral}	
\end{figure}
%
%
%\renewcommand{\thefigure}{\theenumi}
%
\item The design parameters for construction are:
\label{const:table1}
\\
\solution See Table. \ref{table:table1}. 
%
\begin{table}[ht!]
\centering
%\begin{tabular}{ |p{3cm}|p{3cm}|  }
%\hline
% \multicolumn{2}{|c|}{Initial Input Values.} \\
%\hline
%a & 4\\
%\hline
%b & 3\\
%\hline
%$\phase{(ACB)$ & $90^{\circ}$ \\
%\hline
%\end{tabular}
\input{./tables/inp.tex}
\caption{Quadrilateral ABCD}
\label{table:table1}	
\end{table}

%\item \textbf{Proof}: Finding angular bisector using unit vectors.
%\label{const:proof}
%\\
%\solution: Let the angle between AB and BC be $\theta$ and between $\vec{R}$ and $\vec{BC}$ be $\alpha$.
%\begin{align}
%\vec{R}=\frac{\vec{A}-\vec{B}}{\norm{A-B}}+\frac{\vec{C}-\vec{B}}{\norm{C-B}}
%\\
%\vec{R}\cdot\vec{BC} = \norm{R} \norm{BC} \cos\theta
%\end{align} 
%The resulting equation after simplifying is,
%\begin{align}
%\cos \theta + 1=\sqrt{2+2\cos\theta}\cos\alpha
%\end{align}
%By squaring on both sides
%\begin{align}
%(\cos \theta +1)^2=2+2\cos\theta(\cos\alpha)^2
%\\
%\cos\theta=2\cos^2\alpha-1
%\end{align}
%The above equation is the formula of $\cos 2\theta$
%$\therefore \alpha=\frac{\theta}{2}$
%\item
%	For simplicity, let the greek letter $\alpha = \phase{ B$.  We have the following definitions.
%\begin{equation}
%\label{eq:tri_trig_defs}
%\begin{matrix}
	%\sin \theta = \frac{b}{c} & 	\cos \theta = \frac{a}{c} \\
	%\tan \theta = \frac{c}{a} & \cot \theta = \frac{1}{\tan \theta} \\
	%\csc \theta = \frac{1}{\sin \theta} & \sec \theta = \frac{1}{\cos \theta}
	%\end{equation}
%
\item Find the angular bisectors of each angle in Fig. \ref{fig:quadrilateral}
\label{const:quadrilateral}
\\
%
\solution 
From the given information, the line equation of acute angular bisector of $\phase B$ in vector form is 
%$\triangle ABC$ are 
\begin{align}
\label{eq:constr_a}
\vec{L_1} \implies \myvec{1 & -2}\vec{x}=0
\end{align}
Where the direction ratio of the line $\vec{L_1}$ are obtained by equation \eqref{const:proof})
Vector form of angular bisector of $\phase C$ is
\begin{align}
\label{eq:constr_b}
\vec{L_2} \implies \myvec{0.72 & 1}\vec{x} = 6.48
\end{align}
Vector form of angular bisector of $\phase A$ is
\begin{align}
\label{eq:constr_c}
\vec{L_3} \implies \myvec{1.20 & 1}\vec{x} = 7.6
\end{align}
Vector form of angular bisector of $\phase D$ is
\begin{align}
\label{eq:constr_d}
\vec{L_4} \implies \myvec{-2.4 & 1}\vec{x} = -10.9
\end{align}
\item To find the point of intersection of the angular bisectors?
\solution
$\vec{E}$ is obtained by using line equations $\vec{L_1}$ and $\vec{L_2}$ as matrix equations.
\begin{align}
\myvec{1 & -1\\0.72 & 1}\vec{x}=\myvec{0\\6.48}
\end{align}
The augmented matrix for the above equation is row reduced as follows
\begin{align}
\myvec{1 & -2 & 0\\0.72 & 1 & 6.48} 
\xleftrightarrow {R_2\leftarrow \frac{R_2-0.72R_1}{2.44}}\myvec{1 & -2 & 0\\0 & 1 & 2.65} 
\\
%\myvec{1 & 2 & 4\\2 & 4 & 12} 
\xleftrightarrow {R_1\leftarrow R_1 + 2R_2}\myvec{1 & 0 & 5.3\\0 & 1 & 2.65} 
\label{eq:line_aug}
\\
\implies \vec{x}=\myvec{5.31\\2.65}
\end{align}
%
$\because$ row reduction of the $2\times 3$ matrix
\begin{align}
\myvec{1 & -2 & 0\\0.72 & 1 & 6.48} 
\end{align}
%
results in a matrix with 2 nonzero rows, its rank is 2.  Similarly, the rank of the matrix 
%
\begin{align}
\myvec{1 & -2\\0.72 & 1} 
\end{align}
is 2, from \ref{eq:line_aug}. 
\\
$$\therefore \vec{E}=\myvec{5.3137\\2.6568}$$
\\
$\vec{F}$ is obtained by using line equations $\vec{L_1}$ and $\vec{L_3}$
\begin{align}
\vec{F}=\myvec{4.472\\2.236}
\end{align}
$\vec{G}$ is obtained by equating line equations $\vec{L_3}$ and $\vec{L_4}$
\begin{align}
\vec{G}=\myvec{5.119\\1.460}
\end{align}
$\vec{H}$ is obtained by equating line equations $\vec{L_2}$ and $\vec{L_4}$
\begin{align}
\vec{H}=\myvec{5.545\\2.489}
\end{align}
The values are listed in 
%\item List the  derived values.
%\label{const:table2}
%\\
%\solution See  
Table. \ref{table:table2} 
\begin{table}[ht!]
\centering
%\begin{tabular}{ |p{3cm}|p{3cm}|  }
%\hline
% \multicolumn{2}{|c|}{Derived Values.} \\
%\hline
%$\vec{M}$ & $$\begin{pmatrix}2\\1.5\end{pmatrix}$$\\						
%\hline
%$\vec{D}$ & $$\begin{pmatrix}4\\3\end{pmatrix} $$\\
%\hline
\input{./tables/inp1.tex}
\caption{Cyclic Quadrilateral EFGH}
\label{table:table2}

\end{table}
%
\item Draw Fig. \ref{fig:quadrilateral}.	
\\
\solution The  following Python code generates Fig. \ref{fig:quadri_py}
%
\begin{lstlisting}
codes/quad.py
\end{lstlisting}
\begin{figure}[!ht]
\centering
\includegraphics[width=\columnwidth]{./figs/quad.eps}
\caption{Quadrilateral generated using python}
\label{fig:quadri_py}
\end{figure}
%
and the equivalent latex-tikz code generating Fig. \ref{fig:quadrilateral} is 
\begin{lstlisting}
figs/quad.tex
\end{lstlisting}
%
The above latex code can be compiled as a standalone document as
\begin{lstlisting}
figs/quad_fig.tex
\end{lstlisting}
%
%
%
%
\end{enumerate}
